Given the goal of adding bugs to real-world programs in an automated way, there are a large number of system designs and approaches.
In order to clarify our design for LAVA, in this section we will briefly examine alternatives.

First, one might consider compiling a list of straightforward, local program transformations that reduce the security of the program. 
For example, we could take all instances of the \texttt{strlcpy} and \texttt{strncpy} functions and replace them with the less secure \texttt{strcpy}, or look for calls to \texttt{malloc} and reduce the number of bytes allocated.
This approach is appealing because it is very simple to implement (for example, as an LLVM pass), but it is not a reliable source of bugs. 
There is no reliable way to tell what input (if any) causes the newly buggy code to be reached; and on the other hand, many such transformations will harm the correctness of the program so substantially that it crashes on \emph{every} input.
In our initial testing, transforming instances of \texttt{strncpy} with \texttt{strcpy} in \texttt{bash} just caused it to crash immediately.
The classes of bugs generated by this approach are also fundamentally limited and not representative of bugs in modern programs.

A more sophisticated approach is suggested by Keromytis~\cite{}: targeted symbolic execution could be used to find program paths that are potentially dangerous but currently safe; the symbolic path constraints could then be analyzed and used to remove whatever input checks currently prevent a bug. 
This approach is intuitively promising: it involves minimal changes to a program, and the bugs created would be realistic in the sense that one could imagine them resulting from a programmer forgetting to correctly guard some code. 
However, each bug created this way would come at a high computational cost (for symbolic execution and constraint solving), and would therefore be limited in how deep into the program it could reach. 
This would limit the number of bugs that could be added to a program.

By contrast, the approach taken by LAVA is computationally cheap---its most expensive step is a dynamic taint analysis, which only needs to be done once per input file. 
Each bug is guaranteed to come with a triggering input. 
In our experiments, we demonstrate that even a single input file can yield thousands of bugs spread throughout a complex program such as \texttt{tshark}.
