In this paper, we have introduced LAVA, a fully automated system that can inject large numbers of realistic bugs into C programs.
LAVA has already been used to introduce over 2000 realistic buffer overflows into open-source Linux C programs of between 10,000 and 2 million lines of code.  
The taint-based measures employed by LAVA to identify attacker-controlled data for use in creating new vulnerabilities should be usable to inject other classes of vulnerabilities than the buffer overflow we demonstrate here, and we will pursue that actively.  
We believe LAVA will be of immense value as an on-demand source of ground truth corpora of very large size.
The availability of these corpora should energize research and development of automated vulnerability discovery tools and techniques, as well as the evaluation thereof.


%LAVA is fast, injecting a new buffer overflow into a program like \verb+file+ in less than 20 seconds.

