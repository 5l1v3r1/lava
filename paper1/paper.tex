\documentclass[conference]{IEEEtran} 
\IEEEoverridecommandlockouts
\usepackage{epsfig,graphicx,subcaption}
% \usepackage[hyphens]{url}
\usepackage{hyperref}


%\documentclass[a4paper,10pt]{article}
\usepackage[utf8]{inputenc}
\usepackage{xeCJK}
\setmainfont{Times}
\setCJKmainfont[BoldFont=STHeiti,ItalicFont=STKaiti]{STSong}
\setCJKsansfont[BoldFont=STHeiti]{STXihei}
\setCJKmonofont{STFangsong}

%opening
\title{Repeatable Reverse Engineering with PANDA
  \thanks{This work is sponsored by the Assistant Secretary of Defense
    for Research \& Engineering under Air Force Contract
    \#FA8721-05-C-0002.  Opinions, interpretations, conclusions and
    recommendations are those of the author and are not necessarily
    endorsed by the United States Government.} }

\author{
\IEEEauthorblockN{Brendan Dolan-Gavitt\IEEEauthorrefmark{1}, Josh Hodosh\IEEEauthorrefmark{2}, Patrick Hulin\IEEEauthorrefmark{2}, Tim Leek\IEEEauthorrefmark{2}, Ryan Whelan\IEEEauthorrefmark{2}}
\\
\small (Authors listed alphabetically) \\
\\
\IEEEauthorblockA{\IEEEauthorrefmark{1}NYU\\brendandg@nyu.edu}
\IEEEauthorblockA{\IEEEauthorrefmark{2}MIT Lincoln Laboratory\\
\{josh.hodosh,patrick.hulin,tleek,rwhelan\}@ll.mit.edu}
}

\begin{document}

\maketitle

\begin{abstract}

We present PANDA, an open-source tool that has been purpose-built to support whole
system reverse engineering. It is built upon the
QEMU whole system emulator, and so analyses have access to all code
executing in the guest and all data.  PANDA adds the ability to record
and replay executions, enabling iterative, deep, whole system
analyses.  Further, the replay log files are compact and shareable,
allowing for repeatable experiments.  A nine billion instruction
boot of FreeBSD, e.g., is represented by only a few hundred
MB.  PANDA leverages QEMU's support of thirteen different
CPU architectures to make analyses of those diverse instruction sets possible
within the LLVM IR.  In this way, PANDA can have a single dynamic
taint analysis, for example, that precisely supports many CPUs.  PANDA
analyses are written in a simple plugin architecture which includes a
mechanism to share functionality between plugins, increasing analysis
code re-use and simplifying complex analysis development.  We
demonstrate PANDA's effectiveness via a number of use cases, including
enabling an old but legitimately purchased game to run despite a
lost CD key, in-depth diagnosis of an Internet Explorer crash, and
uncovering the censorship activities and mechanisms of an IM
client.

\end{abstract}

\section{Motivation}
\label{sec:motivation}
\label{sec:motivation}

Bug-finding tools have been an active area of research for almost as long as computer programs have existed. 
Techniques such as abstract interpretation, fuzzing, and symbolic execution with constraint solving have been proposed, developed, and applied.
But evaluation has been a problem, as  ground truth is in extremely short supply.
Vulnerability corpora exist~\cite{Kass:2005} but they are of limited utility and quantity.
These corpora fall into two categories: historic and synthetic.
Corpora built from historic vulnerabilities contain too few examples to be of much use~\cite{Zitser:2004}.
However, these are closest to what we want to have since the bugs are embedded in real code, use real inputs, and are often well annotated with precise information about where the bug manifests itself.
The author's own experience creating such a corpus was that it is a difficult and lengthy process; a corpus of only fourteen very well annotated historic bugs with triggering inputs took about six months to construct. 
In addition, public corpora have the disadvantage of already being released, and thus rapidly become stale.
We can expect tools to have been trained to detect bugs that have been released.
Given the commercial price tag of new exploitable bugs, which is widely understood to begin in the mid five figures~\cite{Tsyrklevich:2015}, it is hard to find real bugs for our corpus that have not already been used to train tools.
And, while synthetic code stocked with bugs, auto-generated by scripts, can provide large numbers of diagnostic examples, each is only a tiny program and the constructions are often considered unrepresentative of real code~\cite{Kratkiewicz:2005,Juliet:2012}.

In practice, a vulnerability discovery tool is typically evaluated by running it and seeing what it finds. 
Thus, one technique is judged superior if it finds more bugs than another.
While this state of affairs is perfectly understandable, given the scarcity of ground truth, it is an obstacle to science and progress in vulnerability discovery.
There is currently no way to measure fundamental figures of merit such as miss and false alarm rate for a bug finding tool.

We propose the following requirements for bugs in a vulnerability corpus, if it is to be useful for research, development, and evaluation.
Bugs must

\begin{enumerate}
\item Be cheap and plentiful
\item Span the execution lifetime of a program
\item Be embedded in representative control and data flow
\item Come with an input that serves as an existence proof 
\item Manifest for a very small fraction of possible inputs
\end {enumerate}

\noindent
The first requirement, if we can meet it, is highly desirable since it enables frequent evaluation and hill climbing. 
Corpora are more valuable if they are essentially disposable. 
The second and third of these requirements stipulate that bugs must be realistic.
The fourth means the bug is demonstrable and serious, and is a precondition for determining exploitability. 
The fifth requirement is crucial.
Consider the converse: if a bug manifests for all or a large fraction of inputs it is trivially discoverable by simply running the program.

The approach we propose is to create a synthetic vulnerability via a few judicious and automated edits to the source code of a real program.
We will detail and give results for an implementation of this approach that satisfies all of the above requirements.
We call this implementation LAVA for Large-scale Automated Vulnerability Addition.    
A serious bug such as a buffer overflow can be injected by LAVA into a program like \verb+file+, which is 13K LOC, in about a 15 seconds.
LAVA bugs manifest all along the execution trace, in all parts of the program, shallow and deep, and make use of mostly completely normal data flow.
By construction, a LAVA bug comes with an input that triggers it, and no other input can have this effect upon the program.


\section{PANDA System}
\label{sec:pandasys}

In this section, we describe the four main novel aspects of PANDA: its
record/replay facility, its plugin architecture, its ability to
use a single analysis for multiple architectures, and its ability to emulate
Android systems.

\input{recordreplay}
\input{plugins}
\input{architectureneutral}
\input{android}


\section{Plugin Details}
\label{sec:plugins}

Here we detail a few specific plugins commonly used when reverse
engineering with PANDA.

\input{tzb}
\input{syscalls}
\input{shadowcallstack}
\input{scissors}
\input{taint}


\section{Case Studies}
\label{sec:usecases}

In this section, we present three compelling RE use cases for PANDA.
In the first, an old PC game for which the CD key has
been lost is rapidly made whole again by locating the key verification
code and harnessing it to produce keys on demand.  In the second, a
Windows Internet Explorer vulnerability is diagnosed in depth from a
whole-system replay, indicating not merely that it is a use-after-free
bug but pointing the finger at a precise CVE number.  In the third, an
IM client suspected of censoring messages is quickly
determined to be doing so via a blacklist which is also readily
extracted.  Note that, while we end up using many of the plugins
mentioned in Section~\ref{sec:plugins}, no attempt was made to cover
all of them with our use cases.  Rather, we allowed the task at hand
to drive the plugins employed.


\input{reversesomeoldtech}
\input{reversesomevuln}
\input{reversesomebadsoftware}
\input{reproduce}

\section{Related Work}
\label{sec:relwork}
\input{relatedwork}


\section{Limitations and Future Work}
\label{sec:future}
\input{limitations}

\section{Conclusion}
In this paper, we have introduced LAVA, a fully automated system that can rapidly inject large numbers of realistic bugs into C programs.
LAVA has already been used to introduce over 4000 realistic buffer overflows into open-source Linux C programs of up to 2 million lines of code.  
We have used LAVA corpora to evaluate the detection powers of state-of-the-art bug finding tools.
The taint-based measures employed by LAVA to identify attacker-controlled data for use in creating new vulnerabilities are powerful and should be usable to inject many and diverse vulnerabilities, but there are likely fundamental limits; LAVA will not be injecting logic errors into programs anytime soon.
Nevertheless, LAVA is ready for immediate use as an on-demand source of realistic ground truth vulnerabilities for classes of serious vulnerabilities that are still abundant in mission-critical code.
It is our hope that LAVA can drive both development and evaluation of tools and techniques for vulnerability discovery.

%LAVA is fast, injecting a new buffer overflow into a program like \verb+file+ in less than 20 seconds.



\bibliographystyle{plain}
\bibliography{biblio}

\end{document}
